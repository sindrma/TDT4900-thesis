%!TEX root=../main.tex
\textbf{Climbing Mont Blanc - Improving System Usability} \\

Climbing Mont Blanc (CMB) is a system for evaluation of programs executed on modern heterogeneous multicores such as the Exynos Octa chips used in eg. Samsung Galaxy S5 and S6 mobile phones, see https://www.ntnu.edu/idi/card/cmb. CMB evaluates both performance and energy efficiency, and provides the possibility of performance ranking lists and online competitions. A first version of the system is available and under trial use. This master thesis project builds on the project work by Sindre Magnussen finished in December 2015, and is focusing  at continuing the improvement of various aspects of the system and its use. \\

\NoIndent{The project involves the following subtasks: }
\begin{enumerate}
\item Fix the main bugs and known issues found during user testing of CMB in November 2015.

\item Improve the system with a new non-blocking database solution.

\item Improve and extend the CMB system's usability features in accordance with the CMB team's priorities.

\item  Conduct a user-experiment to evaluate usability.

\NoIndent{If time permits:}
\item Propose improvements to the existing stability test with a practical solution for simulating users and their submissions, i.e. a synthetic workload.

\item Propose how to improve the how-to information and the existing database of problems by cleaning up, improving, using experience from TDT-4200 and adding new problems.

\item Propose how to implement a discussion forum which allows discussion of each problem and the use of CMB in general.

\item Implement some of the proposed solutions after approval by, and in collaboration with the CMB team.

\end{enumerate}

The master thesis project is part of the EECS Strategic Research project at IME (www.ntnu.edu/ime/eecs). Many of the tasks assume a good collaboration with master student Christian Chavez. \\

This master thesis project is reserved for master student Sindre Magnussen. \\

Main supervisor: Prof. Lasse Natvig \\
Co-supervisor: Assoc. prof Magnus Själander
