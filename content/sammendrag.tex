\section*{\Huge Sammendrag}
$\\[0.5cm]$

Grensene til prosessorer brukt i smarttelefoner, såkalte heterogene multikjerne prosessorer, flyttes hver gang en ny smartelefonmodell lanseres. Dette har ført til økt interesse utenfor mobilmarkedet for denne typen prossesorer, og de har vist seg å være gode kandidater for å bygge energieffektive superdatamaskinarkitekturer. Det Europeiske prosjektet Mont Blanc har som mål å bruke de nevnte mobilprosessorene til å utvikle en superdatamaskin, som konsumerer 15 til 30 ganger mindre energi sammenlignet med tilsvarende arkitekturer. Prosjektet har underveis resultert i flere prototyper som bruker heterogene multikjerne prosessorer, og prototypene har vist lovende resultater for videre utvikling av energieffektive superdatamaskiner. \\

Som en følge av økt smarttelefonytelse, vil også energiforbruket til disse enhetene øke. Det er flere utfordringer knyttet til energieffektivitet i disse systemene som utfordrer både maskinvare- og systemutviklere. For å hjelpe utviklere å lage energieffektive løsninger, startet Lasse Natvig Climbing Mont Blanc-prosjektet. Systemet er inspirert av andre nettbaserte systemer som tilbyr programmeringskonkurranser, såkalte nettbaserte dommere (``Online Judges''). Tanken bak systemet er å la programmerere konkurrere i å utvikle energieffektive løsninger til et bredt spekter av programmeringsoppgaver. Climbing Mont Blanc-systemet er såvidt vi vet det eneste systemet som fokuserer på energieffektiv koding. Tilbakemeldinger fra brukere viser derimot at noen deler av systemet trenger å forbedre brukbarheten. \\

Denne avhandlingen ser på hvordan brukbarheten til visse deler av systemet kan forbedres. Sanntidsnotifikasjoner, oppdateringer av feedback meldinger, en elektronisk oppslagstavle, oppdateringer av brukergrensesnittet og fjerning av kjente feil har blitt integrert inn i versjon to av Climbing Mont Blanc-systemet. Et brukereksperiment påviste at brukere er mer fornøyd med feedback gitt av system versjon to, med en konfidensverdi på 97.2\%. Resultatet av eksperimentet viste også en tendens til at brukere var mer fornøyd med designet og tilgjengelig bruksinformasjon i system versjon to. Kontinuerlige lavterskel brukertester har også blitt gjennomført for a validere brukbarheten til systemet underveis i utviklingen.

\clearpage
