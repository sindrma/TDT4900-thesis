%!TEX root=../main.tex
\chapter{Introduction}
In this chapter, the motivation (\ref{sec:mot}), problem statement interpretation (\ref{sec:rq}) and contributions (\ref{sec:cont}) for this project are explained. It will also outline the structure of the thesis in section \ref{sec:out}.

\section{Motivation}
\label{sec:mot}
The limits of smartphone processors regarding computational power and energy efficiency are pushed for each new smartphone model. As a result, mobile processors or so-called heterogeneous multicores, has gained increasing interest as components in systems outside the mobile market. The \gls{mb} Project \cite{MB} has shown that heterogeneous multicores are potential candidates for building \gls{hpc} systems, as the performance gap between heterogeneous multicores and microprocessors are closing quickly \cite{a:MB:Raj13}. Increasing performance has been a priority in HPC, and HPC systems performance are expected to hit Exascale performance ($10^{18}$ \gls{flops} (exaFLOPS)) within few years \cite{TOP500}. The project goals are to build a new type of supercomputer architecture, reaching Exascale performance using 15 to 30 times less energy compared to other HPC systems. The project has already constructed multiple prototypes by using heterogeneous processors like the Samsung Exynos \cite{EXY} processor among others. The Barcelona Supercomputing Center coordinates the project, and the project has received funding to last until September 2016. \\

Smartphones are becoming more and more computationally powerful for every release. The increased computing power makes the phone consume more energy, which poses enormous challenges regarding energy efficiency both for hardware and software developers. ARM has developed energy aware big.LITTLE technology for their heterogeneous cores, which consist of both high performance and energy efficient processors. The technology uses Global Task Scheduling to assign threads to the most appropriate CPU based on dynamic runtime information \cite{a:ARM:bL}. Other well-known hardware techniques are also massively used, such as Dynamic Voltage and Frequency Scaling. Among software techniques, Task-Based Programming models like OmpSs \cite{a:ompss2013} has gained increasing interest in recent time, and have been used to develop task based programs for heterogeneous architectures \cite{a:Lien2012}. Both research areas are expected to become even more important in the future. \\

There will be a large need for programmers skilled in developing energy efficient code for heterogeneous multicores. Many \glspl{oj} such as PKU Online Judge \cite{PKU} and Kattis \cite{KATTIS} has existed for years, making it possible for programmers to compete, and self-educate in creating efficient code. However, none of the Online Judges we are aware of are known to evaluate and focus on energy efficiency of submitted programs. Noticing the need to focus on energy efficient programming against heterogeneous multicores, Lasse Natvig started the \gls{cmb} project to train programmers in developing energy efficient code for heterogeneous multicores by providing a new Online Judge. The project began in 2014, and Torbjørn Follan and Simen Støa developed the first prototype version during the Spring of 2014 \cite{mt:T&S}. The system uses an Exynos Odroid-XU3 \cite{XU3} to run and measure the energy consumption of submitted programs. This thesis aims to improve further the system and extend with more functionality with a focus on usability. \\

Developing energy efficient code for heterogeneous systems is challenging. One challenge is to develop high-performance code that uses multiple processor cores of different types. Another challenge is to make the same high-performance code energy efficient. The \gls{cmb} project wants to stimulate students and programmers to develop energy efficient high-performance code for heterogeneous multicores. The PKU Online Judge \cite{PKU} had almost a total of 1,3 million submissions last year, and we believe that increasing the number of submissions on the CMB system might help to solve some of the challenges related to programming heterogeneous multicores. \\

\clearpage

\section{Problem Statement Interpretation}
\label{sec:rq}
Three different sub objectives are identified in the problem statement. Usability improvements are the main focus of this thesis and there are several tasks listed in the problem statement that can be interpreted as usability tasks. Further, there are tasks not strictly related to usability and can instead be viewed as general improvements to the system. Finally, the thesis aims to propose a series of improvements and implement them if time permits. The subtasks listed below will, as of the above discussion, either be labeled as \textbf{U} for usability, \textbf{I} for improvements, or \textbf{P} for improvement proposals. As the tasks were already prioritized by the supervisor in the problem statement, the tasks listed in secondary objectives is assumed as less important compared to the main objectives. However, they are of importance to the \gls{cmb} project in general.

\paragraph*{Main objectives} \hfill

\paragraph*{U1} Fix the main bugs and known issues found during user testing of \gls{cmb} in November 2015:
  \begin{itemize}
    \item \textbf{Uploads on OSX}: Does not work and give an uninformative error message.
    \item \textbf{Submissions locked in a running state}: There should be a timeout on submissions containing infinite loops or code which may leave the submission in a inconsistent state.
    \item \textbf{Highscore list bug}: When switching the sort metric, private submissions should still be hidden.
  \end{itemize}

\paragraph*{I1} Change and/or optimize the existing database management system if necessary to handle more frequent user submissions. Early in this thesis we discovered that optimizing the database were not as important at this point as first predicted, as the system is still a prototype and we have few active users on the system. As a result, we did not think that optimizing the database was of importance at this point. However, switching to a more sophisticated \gls{dbms} should still be done.

\paragraph*{U2} Improve and extend the CMB system's usability features in accordance with the CMB team's priorities. The features prioritized by the CMB team, is a combination of some features listed in the Appendix \ref{apdx:backlog} as well needed usability features found throughout the semester:


\paragraph*{U3} Conduct a user-experiment to evaluate system usability.

\paragraph*{Secondary objectives} \hfill

\paragraph*{P1} Propose improvements to the existing stability test with a practical solution for simulating users and their submissions, i.e. a synthetic workload.

\paragraph*{P2}  Propose how to improve the how-to information and the existing database of problems by cleaning up, improving, using experience gained during the system's use in TDT4200 \cite{TDT4200} and adding new problems.

\paragraph*{P3} Propose how to implement a discussion forum which allows discussion of each problem and the use of CMB in general.

\paragraph*{I2} Implement some of the proposed solutions after approval by, and in collaboration with the CMB team. \\

\section{Project Contributions}
\label{sec:cont}
This project wil contribute towards improving the usability of the Climbing Mont Blanc system. System usability is here defined as the ease of use, ease of learning, and logical design of a software system. The definition includes giving sensible feedback when executing actions against the system, as potentially wrong feedback might distract users and either make the system more difficult to use or learn. This thesis also assumes that the usability of a software system is extended if more features are added to the system frontend, since the range of possible actions for users are extended. \\

Usability is a broad term and there is always room for usability improvements in a software system. This thesis is restricted to improve a subset of the usability aspects listed in Appendix \ref{apdx:backlog}, as well as urgent usability aspects discovered throughout this thesis. Users of the system is defined in this thesis as both administrators using the admin interface and regular users using the frontend browser interface (the website \url{https://climb.idi.ntnu.no}). Furthermore, this thesis also aims to further extend the functionality of the system with focus on contributing towards usability. Contributions are summarized in table \ref{}.

%awesome table


\section{Outline}
\label{sec:out}
This report is structured as follows:\\

\noindent
\textbf{Chapter \ref{ch:background}}: Explains the state-of-art. It will present the Mont Blanc project, the project which motivates the Climbing Mont Blanc project. The chapter will also present the Climbing Mont Blanc project, and the system architecture, energy measurement method, the system environment, and system security will be presented. The chapter will also present the extensions planned during the Spring of 2016. Finally, related \glspl{oj} and programming competitions finishes the chapter. \\

There are also many appendices attached to this thesis due to a request from the main supervisor.
