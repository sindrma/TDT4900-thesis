%!TEX root=../main.tex
\chapter{Introduction}
In this chapter, the motivation (\ref{sec:mot}), problem statement interpretation (\ref{sec:rq}) and contributions (\ref{sec:cont}) for this project are explained. It will also outline the structure of the text in section \ref{sec:out}.

\section{Motivation}
\label{sec:mot}
The limits of smart phone processors in terms of computational power and energy effiency are constantly being pushed for each new smart phone model. As a result, mobile processors, or so called Mulit Processor System On Chip (MPSOCs) or heterogenous multicores, has gained increasing interest as components in systems outside the mobile market. The \gls{mb} has shown that are ready to be used in \gls{hpc} as the performance gap between MPSoCs and microprocessors are closing quickly \cite{a:MB:Raj13}. The Barcelone Supercomputing Center coordinate the project, and the project has received funding to last until 2017. The project goal are to build a new type of super computer architecture, using 15 to 30 times less energy compared to other HPC systems. The project has already constructed multiple prototypes using Samsung Exynos \cite{m:Exy} heterogenous processors. \\

Modern mobile phones, or smart phones, are becoming more and more computationally powerful for every release. The increased computing power makes the phone consume more energy, which poses a huge challenges both for software and hardware developers.

\section{Problem Statement Interpretation}
\label{sec:rq}


\paragraph*{Main objectives} \hfill


\paragraph*{Secondary objectives} \hfill



\section{Project Contributions}
\label{sec:cont}


\section{Outline}
\label{sec:out}
This report is structured as follows:\\

\noindent
\textbf{Chapter \ref{ch:bck}}: Explains the state-of-art.
