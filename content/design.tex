%!TEX root=../main.tex
\chapter{Climbing Mont Blanc Usability Goals}
This chapter starts with a definition of usability in \gls{oj} systems in section \ref{sec:usability-def}. The section will define usability in software systems, and show aspects and characteristics in other \gls{ojs} which makes them usable with respect to the usability definition presented. Section \ref{sec:cmb-usability} ends the chapter with a discussion of usability goals for the \gls{cmb} system, aiming to make the system as usable as other acknowledged \gls{ojs}.

\section{Definition of Usability in \gls{oj} Systems}
\label{sec:usability-def}
Usability is defined in the ISO 9241 standard Part 11 as: ``the extent to which a product can be used by specified users to achieve specified goals with effectiveness, efficiency and satisfaction in a specified context of use.'' \cite{ISO1998}. The definition is broad and covers many aspects of a product, or in our case a software system. A lot of research has been done in within software system usability, and literature seems to agree that the following five characteristics describe a usable software system \cite{holzinger2005, ferre2001}; \textit{learnability}, \textit{efficiency}, \textit{user retention over time}, \textit{error rate}, and \textit{satisfaction}. A quick summary of each characteristic found in literature is summarized below.

\paragraph*{Learnability:} The users ability to learn to use the system. This also involves the users ability to gain efficiency in using the system and reach their objectives in a quick manner.

\paragraph*{Efficiency:} Users should be allowed to obtain a high level of productivity when using the system. The usability is improved if the user are able to quickly reach their goals when using the system.

\paragraph*{User Retention Over Time:} The user should be able to return to the system after break from using it, and remember the core functionality of the system. A usable software system makes it easy to return into an efficient state without the need to learn the core functionality of the system anew. The characteristic is also refered to as \textit{memorability}.

\paragraph*{Error Rate:} The number of errors a users makes along the path before reaching their goals. A low error rate among users improves the usability of the system.

\paragraph*{Satisfaction:} The users subjective thoughts about the system as well as making the system pleasant to use. This may involve functionality that is both visible and invisible through the system frontend. \\


The definition of \gls{ojs} as presented in section \ref{sec:related} is fairly simple. However, because of its simplicity, it is fairly important that the \gls{oj} is highly usable when users are actively using the system to solve programming problems. The most time spent on using an \gls{oj} by most users is on reading problem descriptions as well as submitting their code for autmatic judgement. The time spent on reading and submitting is a minimalistic compared to the time spent on developing code, and code devlopment often happen of site in an offline environment if not explicitly offered by the \gls{oj}\footnote{Some of the mentioned \gls{ojs} has web based code editors, such as Kattis \cite{KATTIS} or HackerEarth\cite{HACKEREARTH}}. \\



\section{Climbing Mont Blanc Usability Goals}
\label{sec:cmb-usability}

Ease of use + performance + quick response + feedback

Include design principles of an usable software system. Don Norman + Gestallt principles etc.

What makes a system usable? ISO standard ++.
