%!TEX root=../main.tex
\chapter{Conclusion and Future Work}
\label{ch:conclusion}

\section{Conclusion}
This thesis has improved and added more features to the \gls{cmb} system. The features are mainly usability features with a focus on efficiency, learnability, and satisfiability. As usability is a broad term and there is always room for usability improvements in a software system, only a sub-set of the usability improvements prioritized prior to the start of the thesis and urgent usability aspects discovered throughout this Spring have been implemented. Features implemented in the frontend, server, and backend source code have been described in detail. \\

The frontend views have also had some renovation, such as the removal of unnecessary information, gathering of equal functionality, and updating components with symbols. The improvements aim to make it easier for users to find important information, to navigate easier, and to more clearly state the outcome of interacting with system components. Feedback messages presented to users have also changed in system version two. Users are now presented with colored feedback messages when interacting with the system to make it easier to differentiate between feedback. Submission errors are also displayed in a pop-up message instead of in a designated view, which removes unnecessary navigation stages. A spinner and real-time update a running submission's state are also implemented in the system. \\

Real-time notifications with Socket.io was integrated into the system to enable automatic state updates of submissions. Further use cases of Socket.io and implementation suggestions of these features have briefly been presented. The potential of Socket.io in the system is huge and enables future developers to implement interesting features. \\

Furthermore, solutions to fixes such as enabling uploads for Mac OS X users, automatic Unix file format conversion, cascading-delete of submissions, and cancelation of submissions locking the backend have been described. The \gls{dbms} used has also been switched from SQLite to MySQL, and all previous data present at the development and production server of \gls{cmb} have successfully been moved into the new databases. \\

A user experiment has been conducted to evaluate system usability. The user experiment compared version one developed by Follan and Støa, with system version two developed in this thesis. The results shows that users are, by a 97.2\% probability, more satisfied with feedback given in system version two. There is also a noticeable trend that users are more satisfied with the design and HowTo-page of system version two, and we would need to almost double the amount of participants in both user studies to be 80\% certain that both the design and HowTo-page also has improved. Improvements not tested as part of the user experiment, has been covered using continuous user testing conducted throughout the semester. \\

In conclusion, this project has contributed towards improving system usability and features of the \gls{cmb} system. It has also contributed with proposals on features and implementation details for the future development of the system. The contributions are valuable for the future of the \gls{cmb} system.


\section{Future Work}
\label{sec:future-work}
This section describes possible future extensions which can be made to the system and the project in general. Each extension has a suggested ordering based on priority, and are marked either as A(high), B(medium), or C(low) priority improvements.

\subsection*{Administration}

\paragraph*{A. Problem descriptions should follow ACM ICPC standards:} As mentioned in Sub-\Cref{sub-sec:prop-problems}, problem descriptions should be in the same format as problem descriptions presented by other \glspl{oj}. Follan and Støa also mentioned best practices when adding new problems to the system, and future administrators should follow them. Best practices and a manual for adding problems can be found in Appendix \ref{apdx:problems}.

\paragraph*{A. Thorough testing of added problems:} Problems descriptions should be thoroughly tested on the development server of \gls{cmb} before adding the problem to the production server. This means that the input and output files of a problem to be added should be heavily tested to make sure it covers all edge cases present in the problem description. It is also important that substantial testing is performed either locally or at the development server, as we do not want users to be denied use of the system because of testing.


\subsection*{Development and Testing}
\paragraph*{B. Extra production server for acceptance testing:} As described in Sub-\Cref{sub-sec:prop-stability-test}, a production-like server should be added to more quickly run extensive testing and generate statistics in a production-like environment. The Jenkins pipeline stage should also be added to launch stability and integration tests automatically. The server should be added as it is considered a bad \gls{ci} practice to run tests on the production server, as it may prohibit regular users from using the system.


\paragraph*{C. Add Database Fixtures:} Database fixtures should be added to the system. A fixture is a defined set of data which can be loaded into the database with ease, which enables quick setup before testing the system. Fixtures might make it easier for future developers to start developing, and run tests quickly during local development. A possible Python module which can be used is Flask-Fixtures \cite{FLASK-FIXTURES}, which enables fixture definitions in the \gls{json} format among others. Features and Improvements

\subsection*{Features and Improvements}
\paragraph*{A. Server Side Timeout and Backend Monitoring:} The server should keep track of timers for each submission running in the system as described in Sub-\Cref{sec:disc-server}. The server can keep track of timers for each submission by for example using gevent coroutines \cite{GEVENT} as discussed in the section. The server could then kill submissions running on a backend if a timeout occurs and notify users by emitting events to the respective client who submitted the code. This can be achieved by exploiting Socket.io namespaces and rooms as presented in \Cref{sec:real-time}. Timeout and run-time information can also be sent in real-time to clients to enable users to have up-to-date information about a running submission.

\paragraph*{A. Port Bash Scripts Into Python Scripts:} Bash scripts defined at the server and backend should be ported to Python scripts instead. Debugging Bash scripts executed over \gls{ssh} has proven to be troublesome, and have also proven to be error prone upon code changes. By rewriting all Bash scripts to Python scripts, it should be easier to debug, and it is also simpler to add unit tests for these parts of the system.

\paragraph*{A. Server Side Zip File Check:} The frontend parses the submitted zip and checks for the correct format as described in Sub-\Cref{sub-sec:impr-frontend-bug}. A similar check should be added to the server code to ensure proper storage of files in the file-system, and is especially important if \gls{api} documentation is released or a \gls{cli} is implemented.

\paragraph*{B. Real-Time Queue Position:} The server should give real-time information about submission position to clients who has submissions in the run queue. As described in Sub-\Cref{sec:disc-server}, a complete copy of the queue can be made on every dequeue and the placement of every submission can be emitted using Socket.io. As discussed in the section, this solution might yield high network traffic during heavy system load, and it might be too simple. Another solution is to make a complete copy still, but only send submission positions to those clients with submissions in the queue. Socket.io rooms give a solution to the latter approach.

\paragraph*{B. Multi File Upload:} It should be possible to upload source files directly and preferably multiple files. More advanced upload feature such as an online code editor, which is offered by \glspl{oj} like \cite{KATTIS} and HackerEarth \cite{HACKEREARTH}, should also be considered by future developers.

\paragraph*{C. Low-level Statistics:} The server should keep track of low-level submission information, such as memory usage, chip temperature, and cache usage. Further, noise levels in running time and energy consumption measurements should by reported as well if applicable. This feature requires changes to all parts of the system.

\paragraph*{C. Login via Feide:} It should be possible for NTNU students to login via Feide. \footnote{Feide is a student's digital identity on the Uninett network: \url{https://www.feide.no/}.} Enabling login through Feide would make it simple for students and professors to start using the system in course-related activities.
