%!TEX root=../main.tex
\chapter{Administration}
\label{apdx:problems}

\section{Adding and Hiding Problems}
Login as an admin user either at http://climb-dev.idi.ntnu.no eller http://climb.idid.ntnu.no depending on where the problem are to be added. Access to the admin interface can be requested by contacting the \gls{cmb} team. When logged in, follow the steps below to add the problem:
\begin{enumerate}
  \item Select ``Problem'' in the the topmost main menu.
  \item Click ``Create'' in the sub menu and execute the following steps:
  \begin{itemize}
    \item Insert problem name in the ``Name'' field. Do not use ‘/’ in the problem name!
    \item Add an explanation in the ``Description'' field. This is an HTML field.
    \item Insert a date in the ``Created''-field.
    \item Write the name of the quantity to be optimized in the ``Goodness Name'' field, or leave it empty if not appropriate. Only added if needed by the checker-program (see \Cref{}).
    \item Note! Do not tick off the ``Visible'' check-box yet, wait until all problem data is uploaded (see below).
  \end{itemize}
  \item Insert the new problem by clicking the ``Submit'' button.
  \item Select ``Uploads'' in the top level horizontal menu bar. Notice that a folder with the newly added problem name has been created, where the name is lower case and spaces are replaced with underscores.
  \item Select the newly created folder and then select the ``problemIO''-subfolder.
  \item When in the ``problemIO'' subfolder, upload the following files using the ``Upload File''-button. Notice there is no multi-file upload. The file names must be the following, and all files must be present.
  \begin{itemize}
    \item \textit{input.txt}: Input for the measured test.
    \item \textit{answer.txt}: Correct answer for the measured test.
    \item \textit{small_input.txt}: Input for the small correctness test.
    \item \textit{small_answer.txt}: Correct answer for the small correctness test.
    \item \textit{checker.cpp}: A problem checker written in C++. It will automatically be compiled to a checker executable. The checker should return 0 on success, any other number on failure.
  \end{itemize}
  \item This step can be skipped if the checker is correct. The checker is executed in the following way:
  \begin{lstlisting}[language=sh]
  ./checker input.txt output.txt answer.txt
  \end{lstlisting}
  where \textit{output.txt} is the submitted program’s output. This means the third command line argument is the file name of the output of a submitted program (argv[2]), and can be created by running the program with the arguments present in the \textit{input.txt}-file.
  \item Return to the ``Problem''-tab, then modify the problem by clicking the pencil symbol, and
  check the ``Visible'' check-box. Click ``Submit'' to save the change. The ``Visible'' check-box makes the added problem visible at the frontend.
\end{enumerate}

To hide a problem from the system frontend, the following steps needs to be executed:
\begin{enumerate}
  \item Choose ``Problem'' in the topmost main menu.
  \item To delete the problem press the Recycle Bin symbol, or select the problem with the the checkbox at the lefthand side and press ``With Selected'' in the sub menu and press ``Delete'' in the dropdown menu that appears.
  \item Choose ``Uploads'' in the topmost main menu.
  \item \textbf{Note! Deleting problems is not should not be done, as it may leave dangling submissions to the deleted problem.}
\end{enumerate}

\section{Checker Example, Simple Diff}

\section{Checker Example with Goodness}
