%!TEX root=../main.tex
\chapter{Administration}
\label{apdx:problems}
It should be noted that the information stated in this chapter was created by Follan and Støa \cite{mt:T&S}. The information is repeated in this chapter in a slightly rewritten and compressed form as it was requested by the main supervisor.

\section{Bitbucket, Jenkins and Google Analytics}
The Jenkins server is installed at the development server of \gls{cmb} and can be reached at \url{http://climb-dev.idi.ntnu.no:9000}. The defined pipeline stages and setup information can be found there. The Google Analytics account can be reached at \url{https://www.google.com/analytics}. The git repositories are accessible via Bitbucket at \url{https://bitbucket.org}. Access to the services are granted by the \gls{cmb} team.

\section{Problem Descriptions Best Practices}
It is important that problem descriptions are clear and unambiguous, and that all necessary information is stated in the description so that users clearly understand the problem. It is recommended to follow the following format when creating problem descriptions:
\begin{enumerate}
    \item \textbf{Description}: An overall description of the problem.
    \item \textbf{Input}: Should present the format of the input. It is also common practice to state the number of inputs as the first line in the input.
    \item \textbf{Output}: Should present the format of the output.
    \item \textbf{Example}: Provide an example of inputs and expected outputs. This is valuable, as users can test the example locally before submitting to the system.
\end{enumerate}
Remember that good problem descriptions attracts more users and submissions.

\section{Adding and Hiding Problems}
Login as an admin user either at \url{http://climb-dev.idi.ntnu.no} eller \url{http://climb.idi.ntnu.no} depending on where the problem are to be added. Access to the admin interface can be requested by contacting the \gls{cmb} team. When logged in, follow the steps below to add the problem:
\begin{enumerate}
  \item Select ``Problem'' in the the topmost main menu.
  \item Click ``Create'' in the sub menu and execute the following steps:
  \begin{itemize}
    \item Insert problem name in the ``Name'' field. Do not use ‘/’ in the problem name!
    \item Add an explanation in the ``Description'' field. This is an HTML field.
    \item Insert a date in the ``Created''-field.
    \item Write the name of the quantity to be optimized in the ``Goodness Name'' field, or leave it empty if not appropriate. Only added if needed by the checker-program (see \Cref{sec:checker-advanced}).
    \item Note! Do not tick off the ``Visible'' check-box yet, wait until all problem data is uploaded (see below).
  \end{itemize}
  \item Insert the new problem by clicking the ``Submit'' button.
  \item Select ``Uploads'' in the top level horizontal menu bar. Notice that a folder with the newly added problem name has been created, where the name is lower case and spaces are replaced with underscores.
  \item Select the newly created folder and then select the ``problemIO''-subfolder.
  \item When in the ``problemIO'' subfolder, upload the following files using the ``Upload File''-button. Notice there is no multi-file upload. The file names must be the following, and all files must be present.
  \begin{itemize}
    \item \textit{input.txt}: Input for the measured test.
    \item \textit{answer.txt}: Correct answer for the measured test.
    \item \textit{small\_input.txt}: Input for the small correctness test.
    \item \textit{small\_answer.txt}: Correct answer for the small correctness test.
    \item \textit{checker.cpp}: A problem checker written in C++. It will automatically be compiled to a checker executable. The checker should return 0 on success, any other number on failure.
  \end{itemize}
  \item This step can be skipped if the checker is correct. The checker is executed in the following way:
  \begin{lstlisting}[language=sh]
  ./checker input.txt output.txt answer.txt
  \end{lstlisting}
  where \textit{output.txt} is the submitted program’s output. This means the third command line argument is the file name of the output of a submitted program (argv[2]), and can be created by running the program with the arguments present in the \textit{input.txt} file.
  \item Return to the ``Problem''-tab, then modify the problem by clicking the pencil symbol, and
  check the ``Visible'' check-box. Click ``Submit'' to save the change. The ``Visible'' check-box makes the added problem visible at the frontend.
\end{enumerate}

To hide a problem from the system frontend, the following steps needs to be executed:
\begin{enumerate}
  \item Choose ``Problem'' in the topmost main menu.
  \item Select the problem that is to be hidden.
  \item Tick on the ``Visible'' check-box.
  \item Click ``Submit'' to save changes.
\end{enumerate}
\textbf{Note! Deleting problems should not be done, as it may leave dangling submissions in the database.}

\section{Checker Example, Simple Diff}
\label{sec:checker-simple}
The checker below is used if a simple diff between expected and actual output is needed.
\begin{lstlisting}[language=c++]
#include <stdlib.h>
#include <iostream>
#include <sstream>
using namespace std;

int main(int argc, char* argv[]) {
	if (argc != 4) {
		cerr << "wrong number of arguments!" << endl;
		return -1;
	}
	stringstream ss;
	ss << "wdiff -3 " << argv[2] << " " << argv[3] << " > /dev/null";
	int retval = system(ss.str().c_str());
	if (retval != 0)
		return 1;
	return 0;
}
\end{lstlisting}

\section{Checker Example with Goodness}
\label{sec:checker-advanced}
The checker below is used for the ``Vertex Cover''-problem and shows how to output the goodness value ``Cover size'', which is defined when adding the problem to the database.
\begin{lstlisting}[language=c++]
// C++ Vertex cover checker for exercise-3
#include <iostream>
#include <fstream>
#include <limits>
#include <assert.h>
#include <algorithm>
#include <set>
#include <vector>

using namespace std;

// Rudimentary edge structure
struct Edge {
    int u, v;
};

// Graph structure
struct Graph {
    // Number of vertices (v) and edges (e)
    int v, e;
    // A vector of edges
    std::vector<Edge> edges;
    // A vector of vertex ids
    std::vector<int> vertices;
};

// Struct for keeping track of a vertex cover and its quality
struct Solution {
    // A vertex cover and associated fitness value
    std::vector<int> cover;
    double fitness;
};

// Checks to see if the argument solution is a valid vertex cover
bool validVertexCover(Graph graph, Solution current) { ... }

// Driver program
int main(int argc, char* argv[]) {
    // Check argc
    if (argc < 3) {
        cout << "Give input filename and solution filename!" << endl;
        return -1;
    }
    // Open input file
    ifstream inputFile(argv[1]);
    if (!inputFile) {
        cout << "No input file found" << endl;
        return -1;
    }
    // Open solution file
    ifstream solutionFile(argv[2]);
    if (!solutionFile) {
        cout << "No solution file found" << endl;
        return -1;
    }
    string code;
    int c, u, v, N, M;
    Graph graph;

    while (inputFile >> code) {
        if (!code.compare("c")) {  // Skip comments
            inputFile.ignore(numeric_limits<streamsize>::max(), '\n');
            continue;
        } else if (!code.compare("p")) {
            // Read number of vertices into N and number of edges into M
            inputFile >> N >> M;

            graph.v = N;
            graph.e = M;

            continue;
        } else if (!code.compare("v")) {
            inputFile >> c;

            graph.vertices.push_back(c);

            continue;
        } else if (!code.compare("a")) {
            inputFile >> u >> v;

            Edge edge = {u, v};
            graph.edges.push_back(edge);

            continue;
        } else {
            // Should never get here with the correct input
            assert(false);
        }
    }
    // Check student solution //
    // Validate header
    string header;
    solutionFile >> header;
    assert(!header.compare("s"));
    // Read in list of vertices and validate that it is a vertex cover
    std::vector<int> vertices;
    int vertexId;
    while (solutionFile >> vertexId) {
        vertices.push_back(vertexId);
    }
    Solution candidate = {vertices, 0.0};
    assert(validVertexCover(graph, candidate));

    // Output for CMB
    cout << "OK" << endl;
    // Goodness value ("Cover") is second output from the checker
    cout << candidate.cover.size() << endl;
    return 0;
}
\end{lstlisting}
